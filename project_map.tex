\documentclass[10pt, a4paper]{article}

\usepackage{cmap} 
\usepackage[T2A]{fontenc}
\usepackage[utf8x]{inputenc}
\usepackage[english]{babel}
\usepackage{amssymb, amsmath}
\usepackage{amscd}
\usepackage{graphicx}
\usepackage[usenames]{color}
\usepackage{braket}
\usepackage{mathrsfs}
\usepackage{enumerate}
\usepackage{xcolor}
\usepackage{hyperref}
\usepackage{indentfirst}
\usepackage{soul}

\def\setZ{{\mathbb{Z}}}
\def\taumax{{\tau_{\mathrm{max}}}}
\def\tr{{\mathrm{tr}}}
\def\rhoin{{\rho_{\mathrm{in}}}}
\def\rhofin{{\rho_{\mathrm{fin}}}}
\def\blue{\textcolor{blue}}
\def\red{\textcolor{red}}
\def\green{\textcolor{green}}
\def\yellow{\textcolor{yellow}}
\usepackage[left=2cm,right=2cm,top=2cm,bottom=2cm ,bindingoffset=0cm]{geometry}


\begin{document}
\section*{The physics of Trotterization in VQEs: the roadmap\\
\small{Yaroslav Herasymenko, for Tobias Gobel}}

\subsection*{An informal summary}

We investigate the VQE ansatz structures and their efficiency, specifically UCC-based ansatzes. We consider the following ingredients as key: the ansatz generators and the Trotterization scheme. These are most basic, and may be crucial for useful quantum advantage in the near future. Still, much is not known about how to do these choices in a best way.

The generator choice in standard UCC theory is based on the perturbation theory. This can be extended to the Trotterized VQEs directly, essentially irrespectively of Trotterization scheme. With this principle, we build a Python module that finds most relevant generators, given the model. It does so up to a certain order in PT, determined by the number of generators requested.

With generators assignment automated, we focus on studying the Trotterization schemes. We identify the two key principles that may be followed in such schemes: going up to higher-order generators (QCA) and recycling the lower-order ones (TUCC*). These choices are found to result in drastically different performances, with one or another being strongly preferred depending on the system. With a combination of analytical and numerical analysis, we identify the physics behind these differences. Based on this analysis, we provide (a) a system-adapted criterion for an efficient Trotterization choice, (b) a new, hybrid principle for the Trotterization choice, and (c) a special tuned-coupling variational scheme.

\subsection*{The setup}


\subsubsection*{Physical systems}

The system Hamiltonians are always functions of coupling $J$ and are of the form:

\begin{equation}
H(J)=(1-J)~H_0+J~H_c,~~~H_0=-\sum_i Z_i,
\end{equation}

for some coupling Hamiltonian $H_c$ which characterizes the system at hand.

The main parameters of $H_c$ are as follows: size+locality, the coupling strength and the presence/absence of a gap. They define the applicability of PT, and influence the required complexity of the VQE ansatz. PT applicability is the crucial feature, as (at least theoretically) it defines the relevance of UCC.

We also have a hypothesis about the fourth, more subtle feature, that may give a less naive condition for the applicability of UCC. Namely, whether the system is `trivially gapped'. That is, if one can change $J$ from its target value to $J=0$ without closing the gap along the way. This may be satisfied even at large coupling, where PT is not directly gapped. But also note that not every gapped model is trivially gapped.

The models we are currently looking at, are all in 1D and are of the following types:
\begin{itemize}
	\item Transverse-field Ising model (TFIM), $H_c=\sum_i X_i X_{i+1}$
	\item Nontransverse-field Ising model (NTFIM), $H_c=\frac{1}{2}\sum_i (X_i+Z_i) (X_{i+1}+Z_{i+1})$
	\item Anisotropic Heisenberg model (AHM),  $H_c=\frac{1}{2}\sum_i (X_iX_{i+1}+Z_iZ_{i+1})$
\end{itemize}

For each such option, we consider three main options for the value of $J$: $J=0.2,~0.5$ and $0.8$. Among these models, all are essentially gapped, except for TFIM at $J=0.5$. Therefore, TFIM at $J=0.8$ is in the nontrivial phase, but the rest are all in the trivial phase (even at large values of coupling).





\subsubsection*{Ansatzes}

 
 
Each ansatz we consider is associated with some unitary $U(\vec{\theta})$ and is given by a state of the form:
\begin{equation*}
\ket{\Psi(\vec{\theta})}=U(\vec{\theta})\ket{\vec{0}},
\end{equation*}
 
We create ansatzes as hierarchies of small children ansatzes, created from a parent ansatz. For a parent ansatz $U(\vec{\theta})$, the children ansatzes $U'(\vec{\theta}')$ are created by fixing some $\theta_{\alpha}$ in $U$ to zero. The choice of active parameters $\theta'_\alpha$ is typically based on perturbation theory. The number of tunable parameters in each ansatz we will refer to as $D$.

The ansatz structure roughly consists of the following:

\begin{itemize}
\item The \textbf{generators} choice for the parent ansatz
\item The \textbf{hierarchy} structure, which allows to keep only a part of $\vec{\theta}$ nonzero (truncated, `children' ansatzes).
\item The \textbf{Trotterization scheme} of the parent ansatz (how its generators and parameters are used)
\end{itemize}

For the \textbf{generators} $T_{\alpha}$, there exist different choices people tend to use in these ansatz types. In this project, for the sake of uniformity, we fix this choice for all types to match that of QCA. This fits into the paradigm of UCCs as a viable special case. In this context, QCA can be viewed as a version of $K=1$ TUCC.

The same uniformity is assumed in the \textbf{hierarchy} structure: for each ansatz, we use the QCA perturbative hierarchy (in $J$).

We focus on comparing different \textbf{Trotterization schemes}, so this is the point of biggest distinction between our ansatzes. Currently we're looking at the following ansatz types:

\begin{itemize}
\item UCC: $U(\vec{\theta})=\exp(i\sum_{\alpha}\theta_{\alpha}T_\alpha)$

This is the basic choice which one would use for classical numerics. For near-term digital quantum computers, it's not well-suited as it's not Trotterized. However, it's a good reference.
\item TUCC (Trotterized UCC): $U(\vec{\theta})=(\prod_{\alpha}\exp(i\frac{\theta_{\alpha}}{K}T_\alpha))^K$

This typically implies a large $K$. We can use this as a reference to try and identify the `Trotter error' of the approximation. However, this is still not practical as multiple gates are constrained to have the same angle. 
\item TUCC* ('unconstrained' Trotterized UCC): $U(\vec{\theta})=\prod^K_{\mu=1}\prod_{\alpha}\exp(i\theta_{\alpha,\mu}T_\alpha)$

Note the subtlety: unlike with TUCC, the number of parameters $\#(\theta_{\alpha,\mu})$ is the multiple of the number of generators $\#(T_\alpha)$. Specifically, $\#(\theta_{\alpha,\mu})=K\cdot\#(T)$. Therefore, here one focuses on using a few good generators, while increasing $\#(\theta_{\alpha,\mu})$ by \textbf{recycling} the generators, i.e. increasing $K$. As the generators we'll use the ones from QCA hierarchy, e.g. $1$st order PT-associated generators.

\item QCA $U(\vec{\theta})=\prod_{\alpha}\exp(i\theta_{\alpha}T_\alpha)$

Here, $K=1$ so the gates aren't wasted, which is an improvement over TUCC. However, QCA doesn't recycle the low-order generators like TUCC* does. Instead, it \textbf{goes up} by including higher PT orders for generator assignment. The advantage is that the size-extensivity theorems hold for this ansatz. Those are the basis of the warranted UCC performance, and they happen to apply for QCA as well.

\item Any intermediate options between QCA and TUCC*. Basically the two tendencies are to \textbf{go up} in generators or \textbf{recycle} them, and we can try to find a combination that involves both.
\end{itemize}

\subsection*{Results}
\subsubsection*{More important results (in chronological/arbitrary order)}

\begin{itemize}
\item \textit{Ordered and chaotic optimization landscape.} In all ansatzes we consider, for small coupling $J$ and gate count $D$, the optimal angles form regular patterns. We managed to explain this regularity by PT and Campbell-Hausdorff formula. For large $J$, this regularity usually breaks down, with optimal angles drifting astray as we add more gates to the ansatz. In TUCC* this is most extreme, with the angles often taking seemingly random values. 
\item \textit{`Trivially gapped' QCA dominance.} For the `trivially gapped' models even at large J, QCA seems to outperform the rest of the options, including TUCC*. TUCC* seems to enter the `barren plateau' regime. This is confirmed by the chaos in optimal angles which is observed without any performance improvement. We explain this through the tunable-coupling cluster analysis.
\item \textit{A hybrid TUCC*-QCA ansatz.} To `combine the power of non-linearities and perturbation theory', one can use a hybrid QCA-TUCC* ansatz. In many cases this happens to outperform both QCA and TUCC*!
\item \textit{Tuned-coupling optimization} Consider incrementally changing $J$ while finding the optimal angles. This is interesting as a diagnostics tool for the VQE ansatz, showing unusual `phase transitions' in the angle behaviour as a function of coupling. Also this may be used as an alternative variational procedure which may converge faster. Finally, the analytics associated with this procedure, seem to explain the efficiency of QCA with the `trivially gapped' systems.
\end{itemize}

\subsubsection*{Less important results}
\begin{itemize}
\item \textit{"Trotter advantage".} Somewhat surprisingly, TUCC at lower $K$ seems to be better than UCC. This is a consistent pattern for most models and most $K$, with the $K=1$ (QCA) typically doing the best of all $K$. As people prefer TUCC* to TUCC anyway, this curiosity is quite academic and isn't worth a publication on its own (but worth to mention as a side result).

\item The `Trotter advantage' extends further: even the angles optimized for UCC, give better performance when used in TUCC (each divided by $K$, naturally), rather then in UCC . This is extremely weird and needs to be thoroughly checked, if we want to include this.

\item The PT series in TFIM seems to be an asymptotic one (goes worse after it goes better). This statement is either false or famous, so I just put it here for our consideration.
\end{itemize}

\subsection*{Further questions \& tasks}

\subsubsection*{Short-term}
\begin{itemize}

\item Substantiate our hypothesis that the angle landscape of QCA is regular and TUCC* -- exponentially flat. Do so either by looking at the Hessian, or checking how much angles wander around without any progress. In the simplest and most optimistic scenario, the latter effect and the flat Hessian will occur simultaneously.

\item Coupling-adapted optimization. Include the performance plots as well as the angle plots. Expand the simulations to AHM and TFIM.

\item Set up the cluster access.

\item Explore the scope of possible hybrid QCA-TUCC* algorithms: different portions of QCA and TUCC*, TUCC* with the parent generators beyond 1st order PT (2nd order?). 

\item Check the `sharpshooting vs. extensive search in the neighbourhood' hypothesis about QCA and TUCC*, by sampling the states generated by these ansatzes. The minimal version is 10 gates each, with TFIM-associated generators. Classify as the `neighbourhood states' those which have small enough entanglement entropy. Specifically, you can consider average single-qubit entropy, with a threshold set to some well-chosen value between $0$ and $1$ - say, $0.3$. Extend to $2$-qubit entropy if this definition isn't sensitive enough.

\item For Yaroslav: write notes on the `cluster analysis'-type analytics.

\item For Yaroslav: check the PT series code. (a) Check if the TFIM PT series is asymptotic, (b) Understand the code and (c) Check if ordering by the overall magnitude can make more sense than the power of $J^k$.

\end{itemize}

\subsubsection*{Long-term}
\begin{itemize}

\item Circuit depth <-> precision. Find out how to produce a "flattened" version of the ansatz, where multiplication order is such that the circuit depth is minimized. 


\item Get a more thorough numerical confirmation of our general claims - more qubits? more parameters? More ansatzes and more models of the same `types' as those we consider, to check if our claims still apply? / may need cluster time / simple options: going to 2D, considering a range of Heisenberg anisotropies

\item To use leading order PT coefficients as starting angles in QCA, write a routine that calculates leading-order backaction. The result can be used as starting angles for the variational procedure or fixed as the `perturbative part' in a hybrid QCA-TUCC* algorithm.

\item For Yaroslav: maybe \textit{prove} that TUCC* is chaotic (either from `cluster analysis' or the nonlinear mapping)

\end{itemize}

\subsubsection*{Outside of our scope (for now)}
\begin{itemize}

\item Is there a better way to deal with the local minima in the optimization landscape?
\item Is there a better way to deal with the long simulation times? / Set up the cluster simulations
\item In TUCC*, find a way to express the angles in terms of the contributions they produce (inverse nonlinear map). (tools - Taylor series, Campbell-Hausdorff formula). Can we get the TUCC* starting guess using nonlinear approx to the PT?
\item Why do we have a `Trotter advantage'? Are we sure it's not a bug?
\item Can we use Wei-Norman expansion and the results from the `exact UCC' paper, anywhere in our work? Any other `cluster analysis' technique?

\item Can we learn a TUCC* representation of a QCA, somehow directly? (numerically - e.g. by optimization procedure, or analytically - by operator mappings?)

\item Perturbing from the `fully coupled side' ($J=1$), show that a non-trivial gapped phase (like TFIM at large coupling) is also reachable for a proper QCA.

\item Check that the PT coefficients work. Also would be interesting to match PT against the ansatz performance. This however requires a tricky identification between the gate count and the number of PT terms included -- so maybe let's drop it for now.

\item When plotting the TUCC* angles, transform them to sums and differences for fixed generator types. You can try Walsh-Hadamard transform, or the one below which can be argued for by using Campbell-Hausdorff analytics. For the example of gates 1-6-11-16 that we discussed in our call, the transformation would be:
\begin{equation}
\begin{pmatrix}
\tilde{\theta}_1 \\
\tilde{\theta}_6 \\
\tilde{\theta}_{11} \\
\tilde{\theta}_{16} 
\end{pmatrix}=\begin{pmatrix}
1 & 1 & 1 & 1\\
1 & 1 & 1 & -1\\
1 & 1 & -1 & -1 \\
1 & -1 & -1 & -1
\end{pmatrix}
\begin{pmatrix}
\theta_1 \\
\theta_6 \\
\theta_{11} \\
\theta_{16} 
\end{pmatrix}
\end{equation}

Of course, this ladder structure generalizes to any number of angles of interest.

\textit{Some of the recent analysis I made, though, suggests that the linear transformations anyway aren't suitable for finding most of the relevant tendencies. This includes the most basic ones, for instance when comparing a two-fold sequence of two interlayed generators against perturbation theory. By the way, the latter explains why the "repelling angle effect" stays only for one iteration. I'll tell you the details.}

\end{itemize}

\end{document}