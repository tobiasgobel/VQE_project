\documentclass[10pt, a4paper]{article}

\usepackage{cmap} 
\usepackage[T2A]{fontenc}
\usepackage[utf8x]{inputenc}
\usepackage[english]{babel}
\usepackage{amssymb, amsmath}
\usepackage{amscd}
\usepackage{graphicx}
\usepackage[usenames]{color}
\usepackage{braket}
\usepackage{mathrsfs}
\usepackage{enumerate}
\usepackage{xcolor}
\usepackage{hyperref}
\usepackage{indentfirst}
\usepackage{soul}

\def\setZ{{\mathbb{Z}}}
\def\taumax{{\tau_{\mathrm{max}}}}
\def\tr{{\mathrm{tr}}}
\def\rhoin{{\rho_{\mathrm{in}}}}
\def\rhofin{{\rho_{\mathrm{fin}}}}
\def\blue{\textcolor{blue}}
\def\red{\textcolor{red}}
\def\green{\textcolor{green}}
\def\yellow{\textcolor{yellow}}
\usepackage[left=2cm,right=2cm,top=2cm,bottom=2cm ,bindingoffset=0cm]{geometry}


\begin{document}
\section*{The VQE project map\\
\small{Yaroslav Herasymenko, for Tobias Gobel}}

We're studying the VQE performance of CC-type ansatz hierarchies for $N$-qubit systems. We focus on the ansatz expressability for different problems of interest. 

Each ansatz we consider is associated with some unitary $U(\vec{\theta})$ and is given by a state of the form:
\begin{equation*}
\ket{\Psi(\vec{\theta})}=U(\vec{\theta})\ket{\vec{0}},
\end{equation*}

We create ansatzes as hierarchies of children ansatzes, created from a parent ansatz. For a parent ansatz $U(\vec{\theta})$, the children ansatzes $U'(\vec{\theta}')$ are created by fixing some $\theta_{\alpha}$ in $U$ to zero. The number of tunable parameters in each ansatz we will refer to as $D$.

The system Hamiltonians are always functions of coupling $J$ and are of the form:

\begin{equation}
H(J)=(1-J)~H_0+J~H_c,~~~H_0=-\sum_i Z_i,
\end{equation}

for some coupling Hamiltonian $H_c$ which characterizes the system at hand.

\subsection*{The setup}

\subsubsection*{Ansatzes}

The ansatz structure roughly consists of the following:

\begin{itemize}
\item The \textbf{type} of the parent ansatz (how its generators and parameters are used)
\item The \textbf{generators} choice for the parent ansatz
\item The \textbf{hierarchy} structure, which allows to keep only a part of $\vec{\theta}$ nonzero (truncated, `children' ansatzes).
\end{itemize}

We focus on comparing different ansatz \textbf{types}, so this is the point of biggest distinction between the structures we consider. Currently we're looking at the following ansatz types:

\begin{itemize}
\item UCC: $U(\vec{\theta})=\exp(i\sum_{\alpha}\theta_{\alpha}T_\alpha)$
\item TUCC (Trotterized UCC): $U(\vec{\theta})=(\prod_{\alpha}\exp(i\frac{\theta_{\alpha}}{K}T_\alpha))^K$, typically implying a large $K$
\item TUCC* ('unconstrained' Trotterized UCC): $U(\vec{\theta})=\prod_{\mu}\prod_{\alpha}\exp(i\theta_{\alpha,\mu}T_\alpha)$
\item QCA $U(\vec{\theta})=\prod_{\alpha}\exp(i\theta_{\alpha}T_\alpha)$
\end{itemize}

For the \textbf{generators} $T_{\alpha}$, there exist different choices people tend to use in these ansatz types. In this project, for the sake of uniformity, we fix this choice for all types to match that of QCA. This fits into the paradigm of UCCs as a viable special case. In this context, QCA can be viewed as a version of $K=1$ TUCC.

The same uniformity is assumed in the \textbf{hierarchy} structure: for each ansatz, we use the QCA perturbative hierarchy (in $J$). Note the subtlety with TUCC*: unlike with TUCC, its `number of Trotter steps' increases the number of parameters. This means that there's no one-to-one correspondence between the generators of \textit{parent} TUCC* and QCA. One needs to take most meaningful generators when defining TUCC*, instead. Most of the time, we use the generators associated to QCA hierarchy, for instance $1$st order PT-associated generators. These are most local and most important, so make TUCC* a strong candidate.

\subsubsection*{Physical systems}

In physical systems, the main parameters are: size+locality, the coupling strength and the presence/absence of a gap. The first one defines the simplicity of the ansatz and the applicability of PT, the second and the third define the (naive) applicability of PT. PT applicability is the crucial feature, and (at least theoretically) it defines the relevance of UCC.

We also have a hypothesis about the fourth, more subtle feature, that may give a less naive condition for the applicability of PT. Let's call it `phase triviality'. That is, the ability to move $J$ from its target value to $J=0$ without closing the gap along the way. Even if the model is gapped, it may well not satisfy this more refined property. At the same time, even at large coupling, the model may be both gapped - and in a trivial phase.

The models we are currently looking at, are all in 1D and are of the following types:
\begin{itemize}
\item Transverse-field Ising model (TFIM), $H_c=\sum_i X_i X_{i+1}$
\item Nontransverse-field Ising model (NTFIM), $H_c=\frac{1}{2}\sum_i (X_i+Z_i) (X_{i+1}+Z_{i+1})$
\item Anisotropic Heisenberg model (AHM),  $H_c=\frac{1}{2}\sum_i (X_iX_{i+1}+Z_iZ_{i+1})$
\end{itemize}

For each such option, we consider three main options for the value of $J$: $J=0.2,~0.5$ and $0.8$. Among these models, all are essentially gapped, except for TFIM at $J=0.5$. Therefore, TFIM at $J=0.8$ is in the nontrivial phase, but the rest are all in the trivial phase (even at large values of coupling).

\subsection*{Results}
\subsubsection*{More important results (in chronological/arbitrary order)}

\begin{itemize}
\item \textit{Ordered and chaotic optimization landscape.} In all ansatzes we consider, for small coupling $J$ and gate count $D$, the optimal angles take values that form a regular pattern. This regularity can typically be explained by PT and Campbell-Hausdorff formula. For large $J$, this regularity usually breaks down, with optimal angles drifting astray as we add more gates to the ansatz. In TUCC* this is most extreme, with the angles often taking seemingly random values. 
\item \textit{`Trivial phase' QCA dominance.} For the `trivial phase' models even at large J, QCA seems to outperform the rest of the options, including TUCC*. TUCC* seems to enter the `barren plateau' regime. This is confirmed by the chaos in optimal angles. We explain this through the cluster analysis idea.
\item \textit{A hybrid TUCC*-QCA ansatz.} To `combine the power of non-linearities and perturbation theory', one can use a hybrid QCA-TUCC* ansatz. In many cases this happens to outperform both QCA and TUCC*!
\end{itemize}

\subsubsection*{Less important results}
\begin{itemize}
\item \textit{"Trotter advantage".} TUCC at lower $K$ seems to be better than UCC. This is a consistent pattern for most models and most $K$, with the $K=1$ (QCA) typically doing the best of all $K$. As people prefer TUCC* to TUCC anyway, this curiosity is quite academic and isn't worth a publication on its own (but worth to mention as a side result).

\item The `Trotter advantage' extends further: even the angles optimized for UCC, give better performance when used in TUCC (each divided by $K$, naturally), rather then in UCC .
\end{itemize}

\subsection*{Further questions \& tasks}

\subsubsection*{Short-term}
\begin{itemize}
\item Extend the numerical experiments to NTFIM (as defined above, (X+Z)(X+Z) model)

\item Check that the PT coefficients work: plot the energy of the PT state, as a function of the number of terms included in PT. 

\textit{Also would be interesting to match against the ansatz performance. This however requires a tricky identification between the gate count and the number of PT terms included -- so maybe let's drop it for now.}

\item Circuit depth <-> precision. Find out how to produce a "flattened" version of the ansatz, where multiplication order is such that the circuit depth is minimized. 

\item To confirm our `barren plateaus' hypothesis about AHM J=0.8 behaviour of TUCC*, let's try to exclude the option that it gets lost in obscure regions of the landscape. Maybe, with zero starting angles it will do better? Let's try this as an alternative optimization routine for TUCC*, with starting angles set to zero independent of the gate count. Should probably do worse than what we do now, but just to be sure.

\item Set up the cluster access.

\item When plotting the TUCC* angles, transform them to sums and differences for fixed generator types. You can try Walsh-Hadamard transform, or the one below which can be argued for by using Campbell-Hausdorff analytics. For the example of gates 1-6-11-16 that we discussed in our call, the transformation would be:
\begin{equation}
\begin{pmatrix}
\tilde{\theta}_1 \\
\tilde{\theta}_6 \\
\tilde{\theta}_{11} \\
\tilde{\theta}_{16} 
\end{pmatrix}=\begin{pmatrix}
1 & 1 & 1 & 1\\
1 & 1 & 1 & -1\\
1 & 1 & -1 & -1 \\
1 & -1 & -1 & -1
\end{pmatrix}
\begin{pmatrix}
\theta_1 \\
\theta_6 \\
\theta_{11} \\
\theta_{16} 
\end{pmatrix}
\end{equation}

Of course, this ladder structure generalizes to any number of angles of interest.

\textit{Some of the recent analysis I made, though, suggests that the linear transformations anyway aren't suitable for finding most of the relevant tendencies. This includes the most basic ones, for instance when comparing a two-fold sequence of two interlayed generators against perturbation theory. By the way, the latter explains why the "repelling angle effect" stays only for one iteration. I'll tell you the details.}

\end{itemize}

\subsubsection*{Long-term}
\begin{itemize}

\item Explore the scope of possible hybrid QCA-TUCC* algorithms: different portions of QCA and TUCC*, TUCC* with the parent generators beyond 1st order PT (2nd order?).

\item To use leading order PT coefficients as starting angles in QCA, write a routine that calculates leading-order backaction. The result can be used as starting angles for the variational procedure or fixed as the `perturbative part' in a hybrid QCA-TUCC* algorithm.

\item Get a more thorough numerical confirmation of our general claims - more qubits? more parameters? More ansatzes and more models of the same `types' as those we consider, to check if our claims still apply? / may need cluster time

\end{itemize}

\subsubsection*{Outside of our scope (for now)}
\begin{itemize}

\item Is there a better way to deal with the local minima in the optimization landscape?
\item Is there a better way to deal with the long simulation times? / Set up the cluster simulations
\item In TUCC*, find a way to express the angles in terms of the contributions they produce (inverse nonlinear map). (tools - Taylor series, Campbell-Hausdorff formula). Can we get the TUCC* starting guess using nonlinear approx to the PT?
\item Why do we have a `Trotter advantage'? Are we sure it's not a bug?
\item Can we use Wei-Norman expansion and the results from the `exact UCC' paper, anywhere in our work? Any other `cluster analysis' technique?
\item What if we move $J$ while incrementally modifying $\theta_{\alpha}$, as the optimization procedure? Under such procedure, how will angles change with $J$?
\item Can we learn a TUCC* representation of a QCA, somehow directly? (numerically - e.g. by optimization procedure, or analytically - by operator mappings?)

\item Perturbing from the `fully coupled side' ($J=1$), show that a non-trivial gapped phase (like TFIM at large coupling) is also reachable for a proper QCA.

\end{itemize}

\end{document}