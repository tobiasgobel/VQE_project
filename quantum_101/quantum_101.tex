\documentclass[11pt, a4paper, tightenlines, notitlepage]{revtex4-1}

\usepackage{cmap} 
\usepackage[T2A]{fontenc}
\usepackage[utf8x]{inputenc}
\usepackage[english]{babel}
\usepackage{amssymb, amsmath}
\usepackage{amscd}
\usepackage{graphicx}
\usepackage[usenames]{color}
\usepackage{braket}
\usepackage{mathrsfs}
\usepackage{enumerate}
\usepackage{xcolor}

\def\setZ{\mathbb{Z}}
\def\blue{\textcolor{blue}}
\def\red{\textcolor{red}}
\def\green{\textcolor{green}}
\def\yellow{\textcolor{yellow}}
\usepackage[left=2cm,right=2cm,top=2cm,bottom=2cm ,bindingoffset=0cm]{geometry}


\begin{document}
\section*{Week 1: single qubit states}
\subsection*{The questions}

\begin{enumerate}

\item What is a qubit?

\item Describe the qubit's Hilbert space (the structure of the state vector, i.e. the "wavefunction"). How many (physically relevant) degrees of freedom does the qubit's state have?

\item Consider the so-called Pauli matrices $\sigma_x$, $\sigma_y$, $\sigma_z$. For a spin-$1/2$ system, find the quantum states which are the eigenstates of these Pauli matrices. What is the physical meaning of such quantum states (google it)? 

\item To describe the physical behaviour of the qubit, one needs to give the Hamiltonian for it. What would be the general form of such a Hamiltonian? How many degrees of freedom (d.o.f.) does it have? How to express it with the Pauli matrices?

\item Consider the hermitian operator $\vec{n}\cdot\vec{\sigma}\equiv n_x\sigma_x+n_y\sigma_y+n_z\sigma_z$, for $n^2_x+n^2_y+n^2_z=1$. What are the eigenvalues of such a combination? What are the eigenstates? How to use these eigenstates to recover the eigenstates of the Hamiltonian parametrized as in the previous exercize?

\item Consider a unitary transformation of a qubit generated by the $\sigma_z, \sigma_y$ Pauli matrices, i.e. $\exp(i\phi\sigma_z)$ and $\exp(i\theta\sigma_y)$. What do they do to an arbitrary qubit state?

\item To properly represent the d.o.f. that the qubit has, one needs to come up with a smart parametrization of the states. Google the Bloch sphere parametrization: how does it work? How to understand it as a sequence of unitary rotations from the previous question, applied to a $\sigma_z$-polarized state?

\item Interprete the unit vector $\vec{n}$ from p.5, as a vector on a sphere with coordinates $\theta,\phi$. How to understand the Bloch sphere states in terms of the eigenstates of the matrix $\vec{n}\cdot\vec{\sigma}$?

\end{enumerate}

\subsection*{The hints}

\begin{enumerate}

\item A two-level quantum system which evolves under external control. (What does this mean? Possible examples?)

\item
\begin{enumerate}\item Hilbert space is the vector space where the states of the system live. The only thing you need is to determine its dimensionality; whether it's a complex or a real vector space; and what's the physical interpretation of its basis vectors (as quantum states).
\item For the degrees of freedom, take into account the normalization condition and the fact that the overall phase is physically irrelevant.
\end{enumerate}

\item To find the eigensystem of a Pauli matrix, solve the characteristic equation. Is there a faster way to do it? (think of the following properties: $\sigma^2_\alpha=1$, $\mathrm{tr}~ \sigma_\alpha=0$)

\item
\begin{enumerate}
\item The Hamiltonian for a quantum system with d-dimensional Hilbert space is a d x d Hermitian matrix. What would be the general form of such a matrix in this case?
\item Express the general Hamiltonian as a linear combination of the three Pauli matrices, and the fourth matrix, namely the identity.	
\end{enumerate} 


\item To find the eigenvalues of $\vec{n}\cdot\vec{\sigma}$, consider the identity $(\vec{n}\cdot\vec{\sigma})^2=1$, $\mathrm{tr}~\vec{n}\cdot\vec{\sigma}=0$. For the eigenstates, solve the characteristic equation. Try to simplify your answers as much as possible, so that they take the neat form.

\item To give a nice expression for the exponential, consider the series for the exponential function, and the property $\sigma^2_\alpha=1$. You can use the outcome to evaluate the action on the qubit state.

\item Try $\exp(i\theta\sigma_y)\exp(i\phi\sigma_z)$ applied to a $\sigma_z$-polarized state. How to relate the answer to the expression that you found in the literature?

\item For this, try to massage the answer for the eigenstates from p.5, in terms of the spherical coordinates of $\vec{n}$, $\theta$ and $\phi$.	
\end{enumerate}

\section*{Week 2: quantum correlations and entanglement}

\subsection*{Some preliminaries}

The Hilbert space of $N$-qubit quantum states is a tensor product of $N$ single-qubit Hilbert spaces. This can be represented as a Kronecker product, which is a space of $2^N$-dimensional vectors with a specific meaning associated to its components. For 2 qubits, the space is $4$-dimensional and the vector components are $\left(c_{00},c_{01},c_{10},c_{11}\right)$. The state will then be:
\begin{equation}
\ket{\Psi}=c_{00}\ket{00}+c_{01}\ket{01}+c_{10}\ket{10}+c_{11}\ket{11}.
\end{equation}

The following notation will be used interchangeably: $\ket{00}$ ($\ket{01}$) and $\ket{0}\otimes\ket{0}$ ($\ket{0}\otimes\ket{1}$). The state $\ket{\Psi}$ is called a tensor product of $\ket{\psi}^A$ and $\ket{\psi}^B$, if its components are related to the components of $\ket{\psi}^{A,B}$ in the following way:
\begin{equation}
\ket{\Psi}\equiv\ket{\psi}^A\otimes\ket{\psi}^B=c^A_{0}c^B_{0}\ket{00}+c^A_{0}c^B_{1}\ket{01}+c^A_{1}c^B_{0}\ket{10}+c^A_{1}c^B_{1}\ket{11}, \textrm{ for: } \ket{\psi}^{A,B}=c^{A,B}_{0}\ket{0}+c^{A,B}_{1}\ket{1}
\end{equation}

The operator $O$ is called to be a tensor product of $O^A$ and $O^B$, if its action on the tensor product states gives a tensor product of actions on the tensor factors:
\begin{equation}
O(\ket{\psi}^A\otimes\ket{\psi}^B)= (O^A\otimes O_B)\ket{\psi}^A\otimes\ket{\psi}^B\equiv (O^A\ket{\psi}^A)\otimes(O^B\ket{\psi}^B)
\end{equation}

\subsection*{The questions}

\begin{enumerate}
\item The Kronecker product form of the tensor product $O^A\otimes O^B$ for qubits $A,B$ is the following 4x4 matrix:
\begin{align}
O^A\otimes O^B=\begin{pmatrix}
(O^A)^0_0\cdot O^B &(O^A)^0_1\cdot O^B\\
(O^A)^1_0\cdot O^B &(O^A)^1_1\cdot O^B
\end{pmatrix}
\end{align}

Check that this is equivalent to the definition of the tensor product above, by checking the specific example of $\sigma_z\otimes\sigma_y$. Note that it is sufficient to check how it acts on the states $\ket{00},\ket{01},\ket{10},\ket{11}$.

\item Show that $(O^A\otimes O^B)(\tilde{O}^A\otimes \tilde{O}^B)=O^A\tilde{O}^A\otimes O^B\tilde{O}^B$ for general operators $O^A$, $O^B$, $\tilde{O}^A$, $\tilde{O}^B$. Then, show that $(\sigma_\alpha\otimes\sigma_\beta)^2=I^A\otimes I^B$.

\item Evaluate the matrix exponent $U(\theta)=e^{i\theta\sigma^A_y\sigma^B_x}$. Evaluate a $\theta$-dependent state: $\ket{\Psi(\theta)}=U(\theta)\ket{00}$. Calculate the following expectation values in the state $\ket{\Psi(\theta)}$: $S_A=\langle\frac{\sigma^A_x+\sigma^A_z}{\sqrt{2}}\rangle$, $S_B=\langle\frac{\sigma^B_x+\sigma^B_z}{\sqrt{2}}\rangle$, $S_{AB}=\langle\frac{\sigma^A_x+\sigma^A_z}{\sqrt{2}}\cdot\frac{\sigma^B_x+\sigma^B_z}{\sqrt{2}}\rangle$. For which states $\ket{\Psi (\theta)}$ does the following hold: $S_{AB}=S_AS_B$?

\item What is a density matrix? How to understand, if the density matrix corresponds to a pure state (check the property $\rho^2=\rho$)? Use Sakurai's book to check this and other basic facts about the density matrices.

Check if the following one-qubit density matrices correspond to pure states: $\rho=\dfrac{1}{2}\begin{pmatrix}
1 & -1\\
-1 & 1
\end{pmatrix}$, $\rho=\dfrac{1}{2}\begin{pmatrix}
1 & -1/2\\
-1/2 & 1
\end{pmatrix}$.

\item What is a partial trace of a density matrix? What is a reduced density matrix of a state? Find a reduced density matrix of a qubit $A$ that comes from the state $\ket{\Psi(\theta)}$ (from the previous exercise). I.e., evaluate $\rho_A(\theta)=\mathrm{tr}_B\left(\ket{\Psi(\theta)}\bra{\Psi(\theta)}\right)$. For which $\theta$ does this reduced density matrix correspond to a pure state?

\item Consider a state $\ket{\Psi}=\frac{1}{2}(\ket{00}+\ket{01}+\ket{10}+\ket{11})$. Calculate the expectation values $S_A$, $S_B$, $S_{AB}$ in this state. Does the following hold: $S_{AB}=S_AS_B$? Evaluate a reduced density matrix in this state, $\rho_A=\mathrm{tr}_B\left(\ket{\Psi}\bra{\Psi}\right)$. Does this density matrix correspond to a pure state?

\item Given the information you obtained about the state $\frac{1}{2}(\ket{00}+\ket{01}+\ket{10}+\ket{11})$, would you conclude that this state is a product or a non-product (entangled) state? If it's a product state, what are the tensor factors of it?


\end{enumerate}

\end{document}